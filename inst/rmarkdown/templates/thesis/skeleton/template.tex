%&latex
% UF Sample ETD Main Document Fall 2016
% Documenting the conversion to xelatex compilation method
% Improved method of handling the single/multiple appendices issue
% Updated font calls to meet latest LaTeX standards
% edited for gatordown following beaverdown as an example, and removed some extra comments for better readability

\documentclass[12pt,final,CPage]{ufthesis} %Use this line for Windows OS

  % Preamble %

  % Added packages/commands to force table environments into single-space
    \usepackage{etoolbox}
    \usepackage{setspace}
    \AtBeginEnvironment{tabular}{\singlespacing}
    \AtBeginEnvironment{lltable}{\singlespacing}
    \AtBeginEnvironment{longtable}{\singlespacing}

  % Add float package and force figures to be placed (H)ere
  \usepackage{float}
  \let\origfigure\figure
  \let\endorigfigure\endfigure
  \renewenvironment{figure}[1][2] {
    \expandafter\origfigure\expandafter[H]
    } {
    \endorigfigure
    }
  % Add package for landscape; should be useful for {longtable} too 
  \usepackage{pdflscape}
  
  % Define Packages To be used and options NOTE: If you add any packages please add them before the hyperref package%
  % here you define all the packages you wish to use in your paper, the ones shown are not all necessary,
% but all have purpose and can be very useful, so leave these as default and add packages as necassary
%\usepackage{graphicx}
%\usepackage[dvipdfmx]{graphicx}
\usepackage{amsmath}
\usepackage{amsthm}
\usepackage{algpseudocode}
%\usepackage{algorithm}
\usepackage{tabularx}
\usepackage{url}
%\usepackage[letterpaper,hmargin=1in,vmargin=1in]{geometry}
\usepackage{lscape}
%\usepackage{hanging}
\usepackage{longtable}
\usepackage{amsfonts}
\usepackage{amssymb}
\usepackage{bm}
%\usepackage[cmss]{sfmath} % Comment this line to use Times New Roman Math Typeface
%\usepackage[cmbright]{sfmath} % Comment this line to use Times New Roman Math Typeface
\usepackage{subfigure}
\usepackage{rotating}
\usepackage{calc}
\usepackage{setspace}
\usepackage{ufenumerate}
%\usepackage{latexsym}
%\usepackage{epsf}
\usepackage{epsfig}
%\usepackage{euscript}
\usepackage[format=hang,justification=raggedright,singlelinecheck=0,labelsep=period]{caption}
%\usepackage[numbers,sort&compress]{natbib} %Use this set-up for numbered reference lists
\usepackage[authoryear]{natbib} %Use this set-up if you want an un-numbered reference list
%\usepackage{hypernat}



\usepackage[hyperfootnotes=false]{hyperref}
%\usepackage[dvipdfmx,hyperfootnotes=false]{hyperref}
%\usepackage[dvips,hyperfootnotes=false]{hyperref}
\hypersetup{colorlinks=true,linkcolor=blue,anchorcolor=blue,citecolor=blue,filecolor=blue,urlcolor=blue,bookmarksnumbered=true,pdfview=FitB} %
% % %DO NOT PLACE ANY PACKAGES AFTER THE HYPERREF SET UP



%\def\UrlFont{\rmfamily} %use this line for Times New Roman
\def\UrlFont{\sffamily} %use this line for CMSS

%\allowdisplaybreaks  % % This command allows equation arrays and similar environments
% % % to break across pages to improve text flow - use only if needed.

% Prevent figures, tables or algorithms from using a separate page or column alone
\renewcommand{\topfraction}{0.85}
\renewcommand{\textfraction}{0.1}
\renewcommand{\floatpagefraction}{0.75}

% Added because of example from brockdown
% This has to do with a default pandoc thing
% http://tex.stackexchange.com/a/258486/77699
\providecommand{\tightlist}{%
  \setlength{\itemsep}{0pt}\setlength{\parskip}{0pt}}

%------------------------------------------%

  % Extra commands or misc formatting such as page alignment or output paper-size commands

%\include{extraparameters}

%------------------------------------------%

  % Set your personal and paper information; examples of each follow the commands
\SetTitle{$title$} % {An Analysis of Something}
\SetAuthor{$author$} % {Joseph A. Student}
\SetDegreeType{$degree$} % {Master of Science}
\SetThesisType{$doctype$} %{Dissertation} %{Thesis}
\SetSubmitdate{$date$} % {January 1, 2013}
\SetGradYear{$commencementyear$} % {2013}
\SetGradMonth{$commencementmonth$} % {October}
\SetDepartment{$department$} % {Research and Evaluation Methodology}
\SetChair{$chair$} % {Some D. Name}
$if(cochair)$
\SetCochair{$cochair$} % {Another P. Name}
\cochairtrue
$endif$
\listtables{$listtables$}
\listfigures{$listfigures$}
\SetBibliography{$bibliography$}
%------------------------------------------%

  % user defined commands in order to generate new commands, macros, and redefine default commands %
  % user defined commands %
% Here is where you define optional commands such as macros, new commands,
% and new environments to be used in your paper

% optional command to prevent a word from breaking across a line %
\hyphenchar\font=-1


% Commands to produce proper bullet list
\newlength{\widthOfItem}
\let\Itemize=\itemize
\let\endItemize=\enditemize
\renewenvironment{itemize}{%
	\begin{Itemize}
		\setlength{\itemsep}{0.5\baselineskip}
		\setlength{\labelwidth}{2em}
		\setlength{\listparindent}{.32in}%
		\setlength{\leftmargin}{.32in}
		\setlength{\rightmargin}{0in}
		\settowidth{\widthOfItem}{\labelitemi}
		\setlength{\labelsep}{\leftmargin-\widthOfItem}
		\renewcommand{\labelitemii}{--}
		\singlespacing}{%
	\end{Itemize}}

% shortcut for setting up inserting \prime command in mathmode to avoid errors %
\newcommand{\p}{^{\prime}}

% shortcuts for prime color text
\newcommand{\red}{\textcolor[rgb]{1.00,0.00,0.00}}
\newcommand{\green}{\textcolor[rgb]{0.00,1.00,0.00}}
\newcommand{\blue}{\textcolor[rgb]{0.00,0.00,1.00}}

% Shorcut commands for mathmatical formulas %

\newcommand{\latex}{\LaTeX 2\ensuremath{\epsilon}}

% THEOREM Environments ---------------------------------------------------
%These environments are provided as a convenience - feel free to modify if needed

\newtheorem{theorem}{Theorem}[chapter]%To link the theorem to each chapter uncomment the chapter option
\newtheorem{lemma}{Lemma}%[theorem]% To link each lemma to a theorem uncomment the theorem option
\newtheorem{corollary}{Corollary}%[theorem]% To link each corollary to a theorem uncomment the theorem option
% to link a corollary to a chapter change the theorem option to chapter
\newtheorem{definition}{Definition}%[chapter] %the same is true for both definitions and assumptions
\newtheorem{assumption}{Assumption}%[chapter] %
\newtheorem{proposition}{Proposition}[chapter]
\newtheorem{algorithm}{Algorithm}[chapter]


%These were some user commands I've run across that I thought some might want to incorporate into their work
%\newcommand{\bdm}{
 %   \begin{displaymath}}

%\newcommand{\edm}{
%    \end{displaymath}}

%\newcommand{\be}{
%    \begin{equation}}

%\newcommand{\ee}{
%    \end{equation}}

%\newcommand{\bea}{
 %   \begin{eqnarray}}

%\newcommand{\eea}{
%    \end{eqnarray}}



  % attempt to control R Markdown syntax highlighting
  $if(highlighting-macros)$
  $highlighting-macros$
  $endif$

%-------------------------------------------------------------------------------------------------------%

  % Begin Main Part of Document %

  \begin{document}

% % % % % % % % % % % % % % % % % % % % % % % % % % % % % % % % % % % % % %
  % Remember - You MUST get a .bst file that matches the Journal in your
% field that you choose as your Reference example
% NONE of these examples will satisfy the Graduate Editorial Office
% if they don't match your Journal example!!!!
% NOTE: If you use a numbered reference system and your references
% are set in parentheses rather than brackets you need to select the
% Natbib option "numbers sort and compress" in the packages.tex file
% % % % % % % % % % % % % % % % % % % % % % % % % % % % % % % % % % % % % %


%Note that the path separator is a forward slash NOT a back slash
%Place YOUR .bst file in the bst folder and use that filename (without the .bst extension)
% as your Bibliography Style file

\bibliographystyle{$bibstyle$}

%-----------------------------------------------------------------------%

\maketitle
% % % % Creates the Title page from the given information
\makecopyright

%------------------------------------------%

$if(dedication)$
	\dedication{$dedication$}
$endif$ % %Includes the dedication - if your dedication is more than a single line
% % % % % % % % % % % % % % % % % %you will need to reduce the vspace amount to keep the text centered verticlly
% % % % % % % % % % % % % % % % % %optional - comment or delete if you are not dedicating to anyone,

%------------------------------------------%

$if(acknowledgements)$
	\acknowledge{$acknowledgements$}
$endif$ % % % %Required - There is no requirement to acknowledge a particular person
% % % % % % % % % % % % % % % % %but you must acknowledge someone (funding source, committee chair, spouse)?

%------------------------------------------%

\tableofcontents %This  creates the Table of Contents, List of Figures, and List of Objects (if any)

$if(listtables)$
  \listoftables %if you have any tables
$endif$

$if(listfigures)$
  \listoffigures
$endif$
% % % % % % % %delete or comment the file you want to remove

%------------------------------------------%

%%This is an optional file. A list of abbreviations is NOT even suggested.
%%Best practice is to define the item the first time it is used in the document

$if(abbreviations)$
	\abbreviations{$abbreviations$}
$endif$


%------------------------------------------%
% This line adds the word CHAPTER to the TOC just before the listing of the chapter and subsections begins
\addtocontents{toc}{\protect\addvspace{10pt}
\noindent{CHAPTER}\protect\hfill\par}{}% This extra line adds the word CHAPTER to the table of contents %
\phantomsection
% original: \include{tex/abstract} %The abstract is created using this file and userinfo.tex
% gatordown edit:
\abstract{$abstract$}

% % % % % % % % % % %If you have a c-chair you must uncomment that line in userinfo.tex AND find the
% % % % % % % % % % %co-chair lines in ufthesis.cls and un-comment those as well

%-----------------------------------------------------------------------%

% This section encompasses the main body of the paper from all the content through to the biographical sketch

% Chapters to be included (more can be added by creating a new chapter#.tex %
% file and then implementing the \inlcude{chapter#.tex} command as seen below %

	$body$

%-----------------------------------------------------------------------%
%
%%%%%% THIS PART ABOUT APPENDICES IS DEPRECIATED, IGNORE %%%%%%
% Use the appropriate file depending upon the number of appendices you have
%
%\include{tex/TwoOrMoreAppendices} %Use this file if you have two or more appendices
%
%\include{tex/OneSingleAppendix} %Use this file if you have one and only one appendix
%
%------------------------------------------%

% Make List of References (BibTeX implemented using the Natbib package)
% un-comment your preferred bibliography style and replace the
% bibliography file "bib/citations" with the name of your .bib file
% REMEMBER!!! If you want un-numbered references comment the Natbib package with
% The numbered options in the packages.tex file and un-comment the package with the authoryear option
% See the included pdfs of the various styles to see the differences.
% The citation style differences are from the \citet{key} and \citep{key} commands
% More options are available; see the Natbib documentation for details

\bibliography{$bibliography$}
\addcontentsline{toc}{chapter}{REFERENCES}
% You can have more than one library of references - put the .bib file
% in the bib folder and call it here
%------------------------------------------%

% Bio Sketch is required and should be in third person, past tense%

\biography{$bio$}

%------------------------------------------%

\end{document}

%-------------------------------------------------------------------------------------------------------%
